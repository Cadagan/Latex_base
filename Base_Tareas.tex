\documentclass[12pt]{article}

% 1 inch margin per side
\usepackage{fullpage}
% Includes graphics
\usepackage{graphicx}
% Packages for Mathematics, by the American Mathematical Society
\usepackage{amssymb}
\usepackage{amsmath}
% Deactivate hyphenation
\usepackage[none]{hyphenat}
% Jump between paragraphs - no indents
\usepackage{parskip}
% UTF-8 and spanish for this template
\usepackage[spanish]{babel}
\usepackage[utf8]{inputenc}
% Links of document
\usepackage{hyperref}
\usepackage{fancyhdr}
\setlength{\headheight}{15.2pt}
\setlength{\headsep}{5pt}
\pagestyle{fancy}

\newcommand{\N}{\mathbb{N}}
\newcommand{\Exp}[1]{\mathcal{E}_{#1}}
\newcommand{\List}[1]{\mathcal{L}_{#1}}
\newcommand{\EN}{\Exp{\N}}
\newcommand{\LN}{\List{\N}}

\newcommand{\comment}[1]{}
\newcommand{\lb}{\\~\\}
\newcommand{\eop}{_{\square}}
\newcommand{\hsig}{\hat{\sigma}}
\newcommand{\ra}{\rightarrow}
\newcommand{\lra}{\leftrightarrow}

% Here you can change my name for yours
\newcommand{\alumno}{Martín Cadagan Guzmán}
% Insert here you task
\newcommand{\task}{Interrogación 2}
% Insert here you course
\newcommand{\course}{IIC2143 - Diseño detallado de software}
% Insert here your University / School name
\newcommand{\university}{Pontificia Universidad Católica de Chile}
% Insert here your faculty/ Dept of studies
\newcommand{\faculty}{Departamento de Ciencias de la Computación}
\rhead{\task - \alumno}

\begin{document}
	\thispagestyle{empty}
	\begin{minipage}{2.3cm}
		\includegraphics[width=2cm]{img/logo.pdf}
		\vspace{0.5cm} 
	\end{minipage}
	\begin{minipage}{\linewidth}
		\textsc{\raggedright \footnotesize
			\university \\
			\faculty \\
			\course \\}
	\end{minipage}
	
	
	
	% Title
	\begin{center}
		\vspace{0.5cm}
		{\huge\bf \task}\\
		\vspace{0.2cm}
		\today\\
		\vspace{0.2cm}
		
		
		\footnotesize{2º semestre 2020 }\\
		\vspace{0.2cm}
		\footnotesize{\alumno}
		\rule{\textwidth}{0.05mm}
	\end{center}
	
	% Start your awnsers from here :)
	\section*{Respuestas}
	
	\subsection*{Pregunta 1}
	\textbf{a)}
	
	% Template for centered figures
	\begin{figure}[h!]
		\centering
		\includegraphics[scale=0.5]{img/logo.pdf}
		\caption{You can change this text for a figure description}
	\end{figure}
	
	% Sample text
	Suponga que D es cerrado y acotado. Si $f$ presenta discontinuidades en el conjunto D, entonces el problema podría igualmente tener soluciones. ¿Cierto o no? ¿Es Chefctio un impostor o no?
	
\end{document}


